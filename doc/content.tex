\maketitle



\chapter{Number Fields}

These notes have been heavily influenced by notes from several sources including: Richard Hill, David Loeffler, Keith Conrad, \cite{marcus},\cite{Samuel} and others.

\section{Recap on rings and fields}
We begin by recalling some basic facts in commutative algebra. Specifically, some ring theory and field theory.	


\begin{rmrk*}
	Throughout, we will not differentiate between $\subset$ and $\subseteq$. If such a distinction needs to be made we will state it or use $\subsetneqq$. 
\end{rmrk*}

\begin{defn}
\label{is_ring}

	A ring $R$ is a set with two binary operations called addition $'+'$ and multiplication $'\cdot'$, such that:
	
	\begin{enumerate}
		\item $R$ is an abelian group with respect to  $+$. Note this means $R$ contains a zero element denoted $0$ and every $r \in R$ has an additive inverse $-r \in R$.
		\item Multiplication is associative and distributive, i.e, \[(xy)z=x(yz) \qquad x(y+z)=xy+xz \qquad (y+z)x=yx+zx\]
	\end{enumerate}
	A ring is called commutative if $xy=yx$ and contains an identity element, denoted $1$. Having a $1$ is sometimes called being unital. Lastly, the subset of elements of $R$ which have a multiplicative inverse are denoted $R^\times$.
	
\end{defn}





%\begin{theorem}
%  \label{comp_open}
%  \lean{comp_open}
%  \uses{comp_cont}
%  \leanok
%  The main theorem.
%\end{theorem}
%\begin{proof}
%  This is the proof.
%\end{proof}
